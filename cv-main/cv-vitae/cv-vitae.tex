%!TEX TS-program = xelatex
%!TEX encoding = UTF-8 Unicode
% Awesome CV LaTeX Template for CV/Resume
%
% This template has been downloaded from:
% https://github.com/posquit0/Awesome-CV
%
% Author:
% Claud D. Park <posquit0.bj@gmail.com>
% http://www.posquit0.com
%
%
% Adapted to be an Rmarkdown template by Mitchell O'Hara-Wild
% 23 November 2018
%
% Template license:
% CC BY-SA 4.0 (https://creativecommons.org/licenses/by-sa/4.0/)
%
%-------------------------------------------------------------------------------
% CONFIGURATIONS
%-------------------------------------------------------------------------------
% A4 paper size by default, use 'letterpaper' for US letter
\documentclass[11pt,a4paper,]{awesome-cv}

% Configure page margins with geometry
\usepackage{geometry}
\geometry{left=1.4cm, top=.8cm, right=1.4cm, bottom=1.8cm, footskip=.5cm}


% Specify the location of the included fonts
\fontdir[fonts/]

% Color for highlights
% Awesome Colors: awesome-emerald, awesome-skyblue, awesome-red, awesome-pink, awesome-orange
%                 awesome-nephritis, awesome-concrete, awesome-darknight

\definecolor{awesome}{HTML}{414141}

% Colors for text
% Uncomment if you would like to specify your own color
% \definecolor{darktext}{HTML}{414141}
% \definecolor{text}{HTML}{333333}
% \definecolor{graytext}{HTML}{5D5D5D}
% \definecolor{lighttext}{HTML}{999999}

% Set false if you don't want to highlight section with awesome color
\setbool{acvSectionColorHighlight}{true}

% If you would like to change the social information separator from a pipe (|) to something else
\renewcommand{\acvHeaderSocialSep}{\quad\textbar\quad}

\def\endfirstpage{\newpage}

%-------------------------------------------------------------------------------
%	PERSONAL INFORMATION
%	Comment any of the lines below if they are not required
%-------------------------------------------------------------------------------
% Available options: circle|rectangle,edge/noedge,left/right

\name{Luis Fernando Delgado Muñoz}{}

\position{Statistical data analyst}
\address{Cali, Colombia}

\mobile{+57 318 834 0741}
\email{\href{mailto:lfdelgadom@unal.edu.co}{\nolinkurl{lfdelgadom@unal.edu.co}}}
\github{lfdelgadom}
\linkedin{lfdelgadom}
\twitter{lfdelgadom}

% \gitlab{gitlab-id}
% \stackoverflow{SO-id}{SO-name}
% \skype{skype-id}
% \reddit{reddit-id}

\quote{I am a dedicated and passionate MSc. Agro-industrial engineer
with expertise in data handling, processing, and analytics of agronomic
and livestock sciences. My professional journey spans as a Data Analyst
in the Cassava Breeding Program at CIAT and a University Professor. I am
driven by my commitment to combining data-driven approaches with
breeding population-based studies of economical traits in agriculture.
As an educator, I focus on empowering students in data handling,
processing, analytics, and gaining insights from data, fostering the
next generation of data-driven leaders. Proficient in R and Python, I
efficiently manipulate and analyze large datasets, develop statistical
models, create compelling visualizations, and effectively communicate
findings to diverse stakeholders.}

\usepackage{booktabs}

\providecommand{\tightlist}{%
	\setlength{\itemsep}{0pt}\setlength{\parskip}{0pt}}

%------------------------------------------------------------------------------



% Pandoc CSL macros
\newlength{\cslhangindent}
\setlength{\cslhangindent}{1.5em}
\newlength{\csllabelwidth}
\setlength{\csllabelwidth}{3em}
\newenvironment{CSLReferences}[3] % #1 hanging-ident, #2 entry spacing
 {% don't indent paragraphs
  \setlength{\parindent}{0pt}
  % turn on hanging indent if param 1 is 1
  \ifodd #1 \everypar{\setlength{\hangindent}{\cslhangindent}}\ignorespaces\fi
  % set entry spacing
  \ifnum #2 > 0
  \setlength{\parskip}{#2\baselineskip}
  \fi
 }%
 {}
\usepackage{calc}
\newcommand{\CSLBlock}[1]{#1\hfill\break}
\newcommand{\CSLLeftMargin}[1]{\parbox[t]{\csllabelwidth}{#1}}
\newcommand{\CSLRightInline}[1]{\parbox[t]{\linewidth - \csllabelwidth}{#1}}
\newcommand{\CSLIndent}[1]{\hspace{\cslhangindent}#1}

\begin{document}

% Print the header with above personal informations
% Give optional argument to change alignment(C: center, L: left, R: right)
\makecvheader

% Print the footer with 3 arguments(<left>, <center>, <right>)
% Leave any of these blank if they are not needed
% 2019-02-14 Chris Umphlett - add flexibility to the document name in footer, rather than have it be static Curriculum Vitae
\makecvfooter
  {July 2023}
    {Luis Fernando Delgado Muñoz~~~·~~~Curriculum Vitae}
  {\thepage~ of \pageref{LastPage}~}


%-------------------------------------------------------------------------------
%	CV/RESUME CONTENT
%	Each section is imported separately, open each file in turn to modify content
%------------------------------------------------------------------------------



\hypertarget{current-appointments}{%
\section{Current Appointments}\label{current-appointments}}

\begin{cventries}
    \cventry{Statistical data analyst - Cassava program. \href{https://alliancebioversityciat.org/who-we-are/luis-fernando-delgado}{\faExternalLink}}{The Alliance Bioversity International and CIAT}{Palmira, Colombia}{Jul 2022--Present}{}\vspace{-4.0mm}
    \cventry{Visiting professor, International Sustainable Agriculture Training Course -\href{https://www.uceva.edu.co/facultades/facultad-de-ingenieria/noticiasfacultadingenieria/pregrado-en-agropecuaria-lidera-curso-internacional-en-formacion-agropecuaria-sostenible/}{R programming module.\faExternalLink}}{Unidad Central del Valle}{Online}{Aug 2022--Oct 2022}{}\vspace{-4.0mm}
    \cventry{Research Assistant Fellow}{The Alliance Bioversity International and CIAT - Manpower group}{Palmira, Colombia}{Dec 2020--Jun 2022}{}\vspace{-4.0mm}
    \cventry{Part time Professor, Engineering faculty, \href{https://www.uceva.edu.co/facultad-de-ingenieria/ingenieria-agropecuaria/}{Agropecuary Engineering program \faExternalLink}}{Unidad Central del Valle.}{Tulúa, Colombia}{Aug 2020--Present}{}\vspace{-4.0mm}
    \cventry{Part time Graduate Professor, Engineering faculty, \href{https://www.palmira.unal.edu.co/DoctCTAlimentos/}{PhD program: Food science ant Technology \faExternalLink}}{Universidad Nacional de Colombia Sede Palmira.}{Palmira, Colombia}{Feb 2020--July 2023}{}\vspace{-4.0mm}
    \cventry{Statistical data analyst, Proyecto Incremento de la competitividad sostenible en la agricultura de ladera en todo el departamento, Valle del Cauca, Occidente.}{Universidad Nacional de Colombia Sede Palmira.}{Palmira, Colombia}{Dec 2021--Feb 2022}{}\vspace{-4.0mm}
\end{cventries}

\hypertarget{education}{%
\section{Education}\label{education}}

\begin{cventries}
    \cventry{Agroindustrial Engineering}{Universidad Nacional de Colombia Sede Palmira}{Palmira, Colombia}{2015}{}\vspace{-4.0mm}
    \cventry{MSc. Agricultural Sciences}{Universidad Nacional de Colombia Sede Palmira}{Palmira, Colombia}{2019}{\begin{cvitems}
\item Best master Thesis Award Laureate (2019)
\item \textbf{Thesis}: Caracterización de oleorresinas de ají Tabasco y Cayenne bajo diferentes niveles de nitrógeno y humedad en el suelo \href{https://repositorio.unal.edu.co/handle/unal/76744}{\faExternalLink}
\end{cvitems}}
\end{cventries}

\hypertarget{research-experience}{%
\section{Research Experience}\label{research-experience}}

\begin{cventries}
    \cventry{Statistical data analyst,  Cassava program}{The Alliance Bioversity International and CIAT}{Palmira, Colombia}{Jul 2022--Present}{\begin{cvitems}
\item Curation and analysis of historical data in the past 50 years from the Cassava Program.
\item Calculating the genetic gain for all the breeding populations the Cassava Program has created.
\item Develop the standard analysis pipelines for yield trial data, quality data and genotypic data.
\end{cvitems}}
    \cventry{Research Assistant Fellow, Quality postharvest lab, Cassava program}{The Alliance Bioversity International and CIAT}{Palmira, Colombia}{Dec 2020--Jun 2022}{\begin{cvitems}
\item Provide support for processing raw roots, textural analysis and optimal cooking time evaluations of breeding populations of cassava program
\item Trained panellist staff on sensory attributes of boiled cassava roots and  constitude the first sensory panel of breeding cassava program.
\item Provide support for data clean, curation, statistical analysis and modelling of RTB Foods project data derived
\end{cvitems}}
    \cventry{Statistical data analysis}{Universidad Nacional de Colombia Sede Palmira}{Palmira, Colombia}{Dec 2021--Feb 2022}{\begin{cvitems}
\item Statistical analysis of Hass avocado demonstration plots in Roldanillo municipality.
\item Statistical analysis of blackberry crop demonstration plots located in the municipality of Pradera
\end{cvitems}}
\end{cventries}

\hypertarget{teaching-and-supervision}{%
\section{Teaching and Supervision}\label{teaching-and-supervision}}

\begin{cventries}
    \cventry{Unidad Central del Valle}{Part time Professor}{Tulúa, Colombia}{Aug 2020--Present}{\begin{cvitems}
\item Engineering faculty, Agropecuary engineering program, courses included: Research tools I and Experimental design with R programming
\end{cvitems}}
    \cventry{Universidad Nacional de Colombia Sede Palmira}{Part time Professor}{Palmira, Colombia}{Feb 2020--Jul 2023}{\begin{cvitems}
\item Engineering and Administration faculty, PhD program: Food science ant Technology, courses included: Analysis and Experimental design
\end{cvitems}}
\end{cventries}

\hypertarget{additional-training-and-professional-development}{%
\section{Additional training and professional
development}\label{additional-training-and-professional-development}}

\begin{cvhonors}
    \cvhonor{}{\textbf{Data Bootcamp for Genomic Prediction in Plant Breeding} (University of Minnesota)}{}{2023}
    \cvhonor{}{\textbf{Genomic Prediction and Selection} (NextGen Cassava Breeding Project)}{}{2022}
    \cvhonor{}{\textbf{Phenotypic modelling of multi-enviroment trials} (Wageningen University)}{}{2021}
    \cvhonor{}{\textbf{Complete Python Developer in 2021: Zero to Mastery} (Udemy)}{}{2021}
    \cvhonor{}{\textbf{Design of experiments for optimization} (Udemy)}{}{2021}
\end{cvhonors}

\hypertarget{skills}{%
\section{Skills}\label{skills}}

\begin{table}[!h]
\centering\begingroup\fontsize{8}{10}\selectfont

\begin{tabular}{lll}
\toprule
\textbf{Analytical} & \textbf{Programming} & \textbf{Software/Tools}\\
\midrule
Phenotype database curation/management & R (advanced) & Git\\
Univariate/multivariate data analysis & Python (intermediate) & GitHub\\
Linear mixed models & Quarto & Shiny\\
Reproducible research & Linux (intermediate) & LaTeX\\
\bottomrule
\end{tabular}
\endgroup{}
\end{table}

\hypertarget{selected-presentations}{%
\section{Selected presentations}\label{selected-presentations}}

\begin{cvhonors}
    \cvhonor{}{\textbf{Agro-industrial trends in the uses of capsicum species}, Invited to speak at the 1 st Ibero-American Congress on Agrifood Sciences, SENA}{}{2017\newline~\newline}
\end{cvhonors}

\hypertarget{publications}{%
\section{Publications}\label{publications}}

\footnotesize

\setlength{\leftskip}{0cm}

\textbf{2023}

\setlength{\leftskip}{1cm}

Meghar, K., Tran, T., \textbf{Delgado, L.F.}, Ospina, M.A., Moreno,
J.L., Luna, J., Londono, L.F., Dufour, D., Davrieux, F.(2022).
\emph{Hyperspectral imaging for the determinationof relevant cooking
quality traits of boiledcassava}.J Sci Food Agric.
\url{https://doi.org/10.1002/jsfa.12654}

\setlength{\leftskip}{0cm}

\textbf{2022}

\setlength{\leftskip}{1cm}

Vásquez Amariles, H., Saavedra Ospina, R., Guerrero Cobos, D., Marín
Beitia, E., Escobar Hernández, G., Cifuentes Gutiérrez, C.A., Caicedo
Vallejo, A.M., Mosquera Escobar, L.E., \textbf{Delgado Muñoz, L.F.}
(2022). \textbf{Opciones tecnológicas para mejorar las prácticas
agronómicas en el cultivo de aguacate Hass en zona de ladera, Colombia:
Parcela demostrativa de aguacate Hass establecida en zona de ladera,
Colombia.} Bogotá: Universidad Nacional de Colombia.
\url{https://repositorio.unal.edu.co/handle/unal/82746}

Vásquez Amariles, H., Guerrero Cobos, D., Castro López, V., Gutiérrez
Pineda O., Gutiérrez Pineda Caicedo Vallejo, A.M., Mosquera Escobar,
L.E., \textbf{Delgado Muñoz, L.F.} (2022). \textbf{Opciones tecnológicas
para mejorar las prácticas agronómicas en el cultivo de mora en zona de
ladera, Colombia: Parcela demostrativa de mora establecida en zona de
ladera.} Universidad Nacional de Colombia.
\url{https://repositorio.unal.edu.co/handle/unal/82722}

\hypertarget{references}{%
\section{References}\label{references}}

\begin{itemize}
\item
  Thierry Tran, PhD. Lead cassava postharvest quality laboratory CIAT.
  Phone: +4450000 ext: 3063 Email:
  \href{mailto:thierry.Tran@cgiar.org}{\nolinkurl{thierry.Tran@cgiar.org}}
\item
  Luis Fernando Londoño, PhD(c), MSc. Senior cassava postharvest quality
  laboratory coordinator CIAT. Phone: +4450000 ext: 3063 Celephone: +57
  310 4315682 Email:
  \href{mailto:L.Londono@cgiar.org}{\nolinkurl{L.Londono@cgiar.org}}
\item
  Daniel Gerardo Cayón Salinas, PhD. Professor Universidad Nacional de
  Colombia Sede Palmira. Phone: +572 2868888 Ext.: 35132-3513 Celephone:
  +57 315 2489251 Email:
  \href{mailto:dgcayons@unal.edu.co}{\nolinkurl{dgcayons@unal.edu.co}}
\end{itemize}


\label{LastPage}~
\end{document}
